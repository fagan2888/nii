
\documentclass[12pt]{article}
%%%%%%%%%%%%%%%%%%%%%%%%%%%%%%%%%%%%%%%%%%%%%%%%%%%%%%%%%%%%%%%%%%%%%%%%%%%%%%%%%%%%%%%%%%%%%%%%%%%%%%%%%%%%%%%%%%%%%%%%%%%%%%%%%%%%%%%%%%%%%%%%%%%%%%%%%%%%%%%%%%%%%%%%%%%%%%%%%%%%%%%%%%%%%%%%%%%%%%%%%%%%%%%%%%%%%%%%%%%%%%%%%%%%%%%%%%%%%%%%%%%%%%%%%%%%
%\usepackage{amsmath}
%\usepackage{amssymb}
\usepackage{graphicx}
%\usepackage{rotating}
\usepackage{chicago}

\renewcommand{\arraystretch}{0.5}
\renewcommand{\baselinestretch}{1.5}
\setlength{\headheight}{0in} \setlength{\headsep}{0in}
\setlength{\oddsidemargin}{0.0in} \setlength{\textheight}{8.5in}
\setlength{\textwidth}{6.5in} \setlength{\topmargin}{0.0in}
\setlength{\footskip}{0.42in} \setlength{\footnotesep}{0.14in}

\begin{document}

\thispagestyle{empty} \normalsize


\title{What Moves Non Interest Income of Banks?}
\author{Rachel Boenigk and Luca Guerrieri} \maketitle

\renewcommand{\baselinestretch}{1} \normalsize

\abstract{When stress tests for the banking sector use a
macroeconomic scenario, an unstated premise is that macro
variables should be useful factors in forecasting the performance
of banks. We assess whether variables such as the ones included in
stress tests for U.S. bank holding companies help improve out of
sample forecasts of chargeoffs on loans, revenues, and capital
measures, relative to forecasting models that exclude a role for
macro factors. Using only public data on bank performance, we find
the macro variables helpful, but not for all measures. Moreover,
even our best-performing models imply bands of uncertainty around
the forecasts so large as to make it challenging to distinguish
the implications of alternative macro scenarios.}

\vspace{1cm}

\noindent JEL Classification Codes: C53, G17, G38.

\noindent KEYWORDS: forecast combinations, macro variables,
measures of banking conditions, stress tests.

\vspace{1cm}

\noindent \footnotesize $^*$ We thank Valentin Bolotnyy, Francisco
Covas, Rochelle Edge, Lutz Kilian, and Andreas Lehnert for useful
comments and discussions. Remaining errors are our own. The views
expressed in this paper are solely the responsibility of the
authors and should not be interpreted as reflecting the views of
the Board of Governors of the Federal Reserve System or of any
other person associated with the Federal Reserve System.
\vspace{0.5cm}


\clearpage \renewcommand{\baselinestretch}{1.5} \normalsize

\section{Introduction}

Stress tests are one of the tools used by banks to understand
their risk exposures. Bank regulators use stress tests to verify
that banks will still be able to maintain adequate levels of
capital under stressful but plausible circumstances.

One type of stress test, referred to as ``scenario analysis,''
involves the application of historical or hypothetical scenarios
to assess the impact of various events on the performance of
banks. Scenario analysis was an integral part of the Comprehensive
Capital Analysis and Review (CCAR), a supervisory assessment of
bank holding companies conducted by the U.S. Federal Reserve in
2011 and 2012, as well as of the Supervisory Capital Assessment
Program (SCAP) in 2009. In these three instances, the 19 largest
U.S. bank holding companies were asked to submit capital plans
extending 9 quarters out and reflecting macroeconomic baseline and
stress scenarios formulated by the Federal Reserve.\footnote{All
bank holding companies with total assets in excess of \$50 billion
were required to participate in CCAR.} As mandated by the
Dodd-Frank Wall Street Reform and Consumer Protection Act,
scenario analysis will continue to be an integral part of stress
tests for the largest bank holding companies in the United
States.\footnote{Other countries have relied on stress tests in
supervising the banking sector. Perhaps most prominently, the
European Banking Authority conducted stress tests based on
scenario analysis throughout the European Union in 2009, 2010, and
2011.}

An unstated premise of stress tests built around a macroeconomic
scenario is that the macro variables should be useful factors in
forecasting the performance of banks. We assess whether variables
such as the ones included in the CCAR scenarios help improve out
of sample forecasts of chargeoffs on loans, revenues, and capital
measures, relative to forecasting models that exclude a role for
macro factors. Furthermore, we construct confidence bands around
the conditional forecasts of our measures of performance for the
banking sector.\footnote{We use a top-down approach that relies on
public data only. Regulators and each bank holding company have
access to a greater wealth of data than what is publicly
disclosed.}

The forecast method that we use is the equal-weighted average of
simple models as first proposed by \citeN{bates1969}. This
approach has been found to yield results near the frontier of best
performance in a varied range of applications, see, for instance
\citeN{stock2004}, who also provide extensive references to the
literature on forecast combinations. \citeN{faust2008} opine that
``its empirical success is part of the folklore of forecasting,''
and like many others include the equal-weighted combination of
simple models as a standard benchmark.

Specifically, we focus on Root Mean Squared Errors (RMSEs) for a
battery of forecast combination models conditional on macro
variables, chosen because they are included in CCAR scenarios, and
a random walk model. Comparisons against a random walk model are
interesting on at least three accounts: 1) the random walk model
beats purely auto-regressive models in RMSE; 2) it has no role for
macro variables; 3) it is a standard benchmark model for forecast
assessment.

The scenarios used in the recent stress tests of bank holding
companies in the United States were not tailored to the business
model of any one specific company. We focus on forecasting
aggregate performance for the companies involved in stress testing
in the United States. It is not our purpose to tailor our
regressions to encompass bank-specific factors. A priori, it is
not clear whether or not bank specific factors would reveal more
robust relations between bank performance measures and macro
factors. We leave that pursuit for further work.

Interest in stress testing, prior to the financial crisis, was
mostly circumscribed to practitioners. Early contributions that
document the use of time series models for stress testing are
those of \citeN{blaschke2001}, \citeN{kalirai2002}, and of
\citeN{bunn2005}. In a recent paper, \citeN{covas2012} model
PPNR's six sub components as well as net chargeoffs to generate a
path of the tier-1 common risk based capital ratio.
\citeN{quagliariello2009}, gathers contributions on the topic of
stress testing from regulators around the world.

A feature common among the early work on stress-testing is the
evaluation of models based on in-sample performance. Like
\citeN{crook2012}, we take an explicit out-of-sample approach.
While there is an established literature that uses financial
factors to facilitate the forecast of macro aggregates, few papers
construct forecasts of financial variables based on macro
factors.\footnote{For the former, see for instance
\citeN{estrella1991} and \citeN{estrella1998}. For the latter, see
\citeN{bellotti2009}.} Our paper is among the few that follow this
latter course. Intuitively, even slow moving variables, like the
unemployment rate, may incorporate useful information for some of
the banking measures that exhibit little high-frequency variation,
such as the tier-1 capital ratio.

The rest of the paper is structured as follows: Section
\ref{Section_modelanddata} spells out the details of the forecast
models used and describes the data; Section \ref{section_results}
lays out our empirical findings; section \ref{section_conclusion}
concludes.


\section{Description of Method} \label{Section_modelanddata}

The measures of performance for the banking sector for our
analysis are aggregates for the top 25 bank holding companies by
total assets assessed quarterly. Using call Report data, the total
assets for these companies amounted to \$9.3 trillion in 2011Q3,
or 74\% percent of the assets of all commercial banks.

We considered measures of performance, from three classes of
financial variables: credit measures, revenue measures, and
capital measures. All measures are derived from the Consolidated
Reports of Condition and Income (Call Report) of the Federal
Deposit Insurance Corporation. We selected net chargeoffs on loans
and leases (chargeoffs for short) as a credit measure, pre
provision net revenue (PPNR) and net interest margin (NIM) as
revenue measures, and the tier-1 regulatory capital ratio as a
capital measure.

To forecast each measure based on macro variables, we use a simple
forecast combination approach. For each macro variable $V^i$ and
for each banking measure, $C$, we estimate a simple regression.
This regression takes the form:
\begin{equation}
C_t=\alpha+ \beta C_{t-1}+\gamma_1 V^i_{t-1}+ \gamma_2 V^i_{t-2}+
\gamma_3 V^{i}_{t-3}+ \gamma_4 V^{i}_{t-4}+u_t
\end{equation}
We forecast out $C$ from regression $i$ according to the scheme
\begin{equation}
C^{F,i}_{t+j}=\alpha+ \beta C^{F,i}_{t+j-1}+\gamma_1 V^i_{t+j-1}+
\gamma_2 V^i_{t+j-2}+ \gamma_3 V^{i}_{t+j-3}+ \gamma_4
V^{i}_{t+j-4}+u_t
\end{equation}
We construct the average forecast of each measure by taking the
unweighted average of the forecasts of that measure across models:
\begin{equation}
C^F_{t+j}= \sum_{i=1}^{i=N} \frac{C^{F,i}_{t+j}}{N}
\end{equation}

The macro variables we consider are: real GDP growth, unemployment
rate (alternatively the change and level), the growth rate of the
national house price index, the term spread, the growth rate of
the S\&P500 index, the implied volatility of the S\&P500 index
options (VIX), and the real interest rate. All of these variables
were included in the baseline and stress scenarios produced by the
Federal Reserve Board as part of the Comprehensive Capital
Analysis and Review in both 2011 and 2012.

We focus on comparing the Root Mean Squared Error (RMSE) for our
forecast combination models and for the forecast implied by a
random walk (i.e., a no change forecast). According to the random
walk model
\begin{equation}
C^F_{t+j} =  C_t.
\end{equation}
In the calculation of the RMSE, we take a pseudo-out-of-sample
approach and keep the estimation window constant at 40 quarters.
Starting with the estimation window from 1986Q1 to 1995Q4
(allowing for a lag length of four quarters) we construct
pseudo-out of sample forecasts going out nine quarters -- the
length of the planning horizon in the U.S. stress test exercise.
We compute RMSEs at each forecast horizon, progressively moving
the estimation window and using an assessment window which spans
1996q1 to 2011Q4. Because we want to mimic the stress testing
process, we take the evolution of the macro variables as observed
in the data. Similarly, within a stress test, the evolution of
macro variables would be implied by a given scenario.

When comparing the RMSE for the random walk and the RMSE for the
forecast combination models, we test for equal predictive ability
using the procedure recommended by \citeN{clark2007}.

\subsection{Data}

The banking data is from the quarterly Consolidated Reports of
Condition and Income (Call Report) that every national, state
member, and insured nonmember bank is required to file on the last
day of each quarter by the Federal Financial Institutions
Examination Council (FFIEC). The Federal Deposit Insurance
Corporation is tasked as the overseer, collecting and reviewing
all submissions. Call Report data used in this analysis are
cleaned and adjusted for bank mergers and acquisitions, using
structure data from the National Information Clearinghouse (NIC)
on mergers and acquisitions.\footnote{ Bank balance sheet
variables are adjusted for mergers between commercial banks by
comparing balance sheet values at the end of the quarter with
those at the beginning of the quarter, accounting for amounts
acquired or lost during the period because of mergers. For
information on the merger-adjustment procedure for income, see the
appendix in English and Nelson (1998).}

Foreign entities are excluded and domestic subsidiaries are
aggregated up to the parent, bank-holding-company (BHC), level. We
aggregate our measures of banking conditions for the top 25 BHCs,
as ranked by total assets, which is assessed quarterly. The
banking data in our regressions start in 1985q1 and end in 2011q3.

Quarterly flows for total net chargeoffs are expressed as a
percentage of total loans and leases; quarterly flows for PPNR are
expressed as a percentage of total assets. The tier-1 regulatory
capital is expressed as a quarterly ratio of risk weighted assets;
and, the net interest margin is expressed as the percentage of net
interest income over interest earning assets.

Aggregation to the level of bank holding company starting with
data for commercial banks may introduce measurement error. As an
alternative to the Call Report data we conduct our analysis using
data from the FR Y-9C Consolidated Financial Statements for Bank
Holding Companies (FR Y-9C). BHCs with total consolidated assets
of \$500 million or more are required to file the FR Y-9C report
on the last day of the quarter. The Federal Reserve acts as the
overseer of this data, collecting, processing and publishing it.
The FR Y-9C report is designed to parallel the Call Report in
terms of the definition of data items. As a result, we are able to
perform our analysis with consistent definitions across measures
of banking conditions. The only deterrent to using FR Y-9C data
instead of Call Report data is that the FR Y-9C data of our
selected banking measures begins six years after the Call Report
data counterpart. We present the results from our analysis using
FR Y-9C data in Section 3.1.

We refer to the macro factors used in the models as two groups of
aggregate factors dubbed ``macro'' and ``financial.'' The macro
group includes the unemployment rate, real GDP growth, and the
growth in the house price index.\footnote{The unemployment rate is
from the Bureau of Labor Statistics; it is a quarterly average of
monthly data. Real GDP is obtained from the National Income and
Products Accounts; it is the log difference of the
chained-weighted index. The house price index is from CoreLogic
and is log-differenced.} The financial group includes a term
spread measure, the growth of the S\&P500 index, the S\&P500
Volatility Index (VIX), and a short-term real interest
rate.\footnote{The term spread measure is the difference between
the yield on a 10 year U.S. government bond and a 3-month U.S.
treasury bill. The S\&P500 index is obtained from Standard\&Poors
and log differenced. The VIX is from the Chicago Board Options
Exchange. Finally the short-term real interest rate is the
interest rate on a three-month U.S. treasury bill.}

Most of the forecast combinations we consider include a forecast
conditional on the unemployment rate. Figure
\ref{figure_unemployment} provides motivation for our interest in
unemployment. The top panel shows percentile curves for the change
in the unemployment rate over a horizon between 1 and 4 quarters.
Remarkably, the bottom panel highlights that periods when the
quarterly change in the unemployment rate is above the 75th
percentile can pick up NBER recessions well ahead of the official
NBER announcements.

\section{Results} \label{section_results}

Figures \ref{figure_chargeoffs} to \ref{figure_tier1} allow a
comparison of the RMSEs for different combinations of simple
models with each figure focusing on a different measure of banking
conditions. We consider eight different models. Model 1, the
broadest, is a combination of forecasts conditional on two groups
of variables, the macro group and the financial group. Model 5
only includes the macro group, while models 2 to 4 are
intermediate models that eliminate one of the financial variables
at a time. Model 6 and 7 pare down the macro variables. Finally,
model 8 considers the performance of a forecast combination that
includes only the financial group. Each figure shows two sets of
results -- we alternatively include the level or the change of the
unemployment rate. In each figure we use a color scheme to
facilitate the comparison of results. The lowest RMSEs at each
horizon are shown against a deep green background, and the highest
RMSEs are shown against a red background. Shades from green to
orange are used for intermediate results.

We determine when macro variables improve upon the random walk
forecast using the test of \citeN{clark2007}. Under the null
hypothesis, the random walk model is the data-generating process.
Then parameters that are zero in population are correctly set to
zero in sample, implying a gain in efficiency.  Conversely, the
alternative model introduces noise into the forecasting process
that inflates its RMSE in sample. Accordingly, \citeN{clark2007}
recommend a downward adjustment of the sample RMSE for the
alternative hypothesis. Thus, it is possible to reject the null of
equal predictive ability even when, in sample, the RMSE of the
alternative hypothesis is higher than the RMSE of the random walk
model.

We use a one sided test. In the tables, the RMSEs for which we
reject the null of equal predictive ability at the 5\%
significance level are highlighted in bold face.

Figure \ref{figure_chargeoffs} focuses on results for total net
chargeoffs. Model 5, with the change in unemployment, real GDP,
and HPI has the lowest RMSE at all horizons and beats a random
walk also at all horizons. Models 6 and 7, both include
unemployment, but drop HPI and GDP in turn.  In both cases, the
combination forecast still beats a random walk forecast, based on
RMSE, if a bit more modestly. In particular, HPI seems to help at
reducing the RMSE at the shorter forecast horizons.

These three models are consistent with the hypothesis that sudden
changes in unemployment can reduce the ability of borrowers to
repay their loans, resulting in substantial increases in
chargeoffs. By contrast, the combination of models that includes
financial variables only (model 8) has the worst performance in
terms of RMSE -- well above a random walk. Moreover, the figure
shows that inclusion of the financial variables substantially
worsens the forecast performance.

Figures \ref{figure_ppnr} and \ref{figure_nim} show results for
our revenue measures, respectively PPNR and NIM. In this case, the
broadest model, Model 1, performs best in terms of RMSE. Model 8,
which had the highest RMSE for total net chargeoffs, displays a
relatively good performance for PPNR and NIM. Overall, however,
even the best-peforming models show more modest gains relative to
a random walk than in the case of total net chargeoffs. In the
case of PPNR, even the best performing model does not beat a
random walk at all horizons.

Figure \ref{figure_tier1} focuses on tier-1 capital. Model 6, with
the level of unemployment and GDP displays the lowest RMSE and
beats the random walk forecast at all horizons. With the tier-1
capital ratio, the forecast combination that has the worst
performance includes financial variables only.

Overall, we were not able to beat a random walk across all
horizons for all of the measures of banking conditions that we
considered. The relative gains in RMSE were most pronounced for
chargeoffs and modest for NIM and tier-1 capital. Figure
\ref{figure_bankingvar} shows the banking measures considered
against the NBER recession dates. Total net chargeoffs show clear
procyclicality. NIM and the tier-1 capital ratio, while much less
volatile, also show some increases in recessions. By contrast,
pre-provision net income does not follow one pattern across the
three recessions spanned by the data available. For instance, in
the most recent recessions, pre-provision net income shows
multiple peaks and troughs.

Even when the forecast combination outperformed the random walk
forecast, the best performing models we could formulate were still
saddled with a substantial degree of forecast uncertainty. As an
example, Figure \ref{figure_forecast} shows forecasts for each of
the measures of banking conditions considered. The estimation
sample ends in 2009Q2, leaving 9 quarters for the assessment
window till the end of our sample in 2011Q3.

Even when we do beat a random walk, the forecast uncertainty bands
in Figure \ref{figure_forecast} imply a striking degree of
uncertainty for each point forecast at different horizons even
when compared to the abnormal variation observed in each series
coinciding with the recent financial crisis. While we cannot claim
to have formulated the most efficient forecast model possible for
each of the measures of banking conditions considered, we
interpret our results as a cautionary factor in the analysis of
capital plans produced by bank holding companies as part of a
stress test exercise.


\subsection{Sensitivity Analysis}

We perform sensitivity analysis regarding several dimensions of
the benchmark forecast exercise. To verify that aggregation of
data for commercial banks in the Call Report dataset to the level
of Bank Holding Company did not skew the benchmark results, we use
alternative data from form FR-Y9C filings that does not require
aggregation. Furthermore, we forecast alternative aggregate
measures for the universe of U.S. commercial banks, instead of
focusing on the largest U.S. bank holding companies.  We consider
a shorter sample that ends before the recent financial crisis.
Finally, we consider sensitivity to alternative choices for the
size of the estimation window -- in turn 60, or 80 quarters,
instead of 40 quarters in the benchmark results. In all cases, to
conserve space, we focus on chargeoffs on loans and leases and do
not report sensitivity results for the additional measures of
banking conditions considered above.

Figure \ref{figure_chgoffs_y9} shows results for the Form Y9C
dataset.\footnote{Our Y9C sample starts in 1997Q1 and ends in
2011Q3.} Figure \ref{figure_chgoffs_allbanks} shows results for an
aggregate measure of chargeoffs for all U.S. commercial banks
using Call Report data. In both cases, model number 5 with
unemployment, GDP, and HPI all in differences remains the best
performing forecast combination and the model that includes all
the variables in our financial group the worst. Moreover, the
performance of the best model still beats that of the random walk
model in terms of lower RMSE. We conclude that neither the
aggregation procedure to bank holding company in the Call Report
dataset, nor consideration of the top 25 bank holding companies
only skews our benchmark results.

Figure \ref{figure_chgoffs_beforecrisis} shows RMSEs based on a
sample of data and assessment window that stop before the recent
financial crisis. The last 40-quarter estimation sample considered
ends in 2005Q3, leaving 9 quarters for the last assessment window
spanning from 2005Q4 to 2007Q4. Even the best forecast combination
model -- still model 5 -- fails to beat a random walk in terms of
RMSE at all of the horizons considered.  The deterioration in
performance is more marked for model number 7, that includes only
the unemployment rate and HPI. We conclude that the inclusion of
HPI brings about an important improvement in performance of the
forecast combination especially when considering a sample that
includes the recent financial crisis.

Finally figures \ref{figure_chgoffs_60quarters} and
\ref{figure_chgoffs_80quarters} show results for a rolling
estimation sample of 60 and 80 quarters, respectively, instead of
40 quarters in the benchmark. We conclude that the main results in
the benchmark experiment continue to hold with these alternative
estimation windows.

\subsection{Sensitivity of Forecasts conditional on CCAR macro scenarios}

In this section we return to the original motivation for our
forecast comparisons, the application to macro stress testing. We
generate forecasts of our four measures of banking conditions
conditional on the macro scenarios included in the most recent
stress test for bank holding companies conducted by the Federal
Reserve, CCAR 2012. Two scenarios were included, a baseline
scenario, and a severe stress scenario. The stress scenario is
meant to represent ``highly adverse conditions'', while the
baseline is meant to capture ``expected economic
conditions.''\footnote{For more information on the design of the
Federal Reserve CCAR 2012 scenarios see \citeN{ccar2012}.}
Considering both scenarios allows us to assess the relative
sensitivity of the banking conditions forecasts to the baseline
and stress CCAR scenarios.

To construct the  forecasts conditional on CCAR scenarios, the
estimation sample ends in 2011q3. Each CCAR scenario includes all
the macro variables needed by the alternative models considered in
the previous sections. The scenarios extend 13 quarters out. The
forecasts we present stop 9 quarters out, as the bank holding
companies in the stress test are only required to produce capital
plans for the next 9 quarters.

Figure \ref{figure_forecast_macroscen} shows dynamic forecasts
conditional on either the stress scenario or the baseline scenario
for each of the measures of banking conditions considered. For
each measure we selected the best performing model out of the 8
models assessed above.\footnote{The models we used for chargeoffs,
PPNR, NIM, and tier-1 capital are, respectively: model 5 with
unemployment in differences, model 1 with unemployment in
differences, model 1 with unemployment in differences, and model 6
with unemployment in levels.} The figure also shows a 2-RMSE
uncertainty band centered around each forecast.

One of the striking features that emerge from Figure
\ref{figure_forecast_macroscen} is that the uncertainty bands for
the baseline scenario encompass the point forecasts conditional on
the stress scenarios. Only in the case of total net chargeoffs and
tier-1 capital ratio do the point forecast veer outside of the
uncertainty bands towards the end of the capital planning horizon.

It is also interesting to consider the sensitivity of the point
forecast to the different scenarios. The difference between the
point forecasts for PPNR and NIM is modest. By contrast, for total
net chargeoffs and tier-1 capital, going 9 quarters out, the
difference between the point forecasts for the baseline and stress
scenarios is sizable. In both cases, it is about half of the
increase observed during the recent financial crisis. However,
notice that Tier-1 capital is predicted to \emph{increase} in a
severe recession. This is in accordance with the pattern observed
in the data. As shown in Figure \ref{figure_bankingvar}, the
tier-1 capital ratio increased during each of the three recessions
for which we have data from the Call Report dataset.

\section{Conclusion} \label{section_conclusion}

This paper contributes to the empirical underpinnings of stress
test exercises. For some but not all measures of aggregate banking
conditions, forecasts conditional on macro variables outperform
random walk forecasts in terms of root mean squared errors. The
largest gains are for total chargeoffs on loans and leases.  We
found relatively more modest gains for net interest income and the
tier-1 capital ratio. However, even our best performing model did
not beat a random walk at all horizons for Pre Provision Net
Income.

Regardless of the gains, we find large RMSEs for the forecasts of
all the measures of banking conditions. The RMSEs are large even
when compared to the large and abnormal variation for each of the
series during the recent financial crisis.

When we apply our preferred forecast models to macro scenarios
used in most recent stress test conducted by the Federal Reserve,
CCAR 2012, we find little sensitivity of the banking measures to
scenarios that are meant to capture large macroeconomic
differences. In all cases, for most of 9-quarter forecast horizon,
the point estimates for the stress scenario are inside the 2-RMSE
uncertainty bands around the forecast conditional on the baseline
scenario.

We cannot claim to have formulated the most efficient forecast
model possible for each of the measures of banking performance
considered. Indeed, we have used only publicly available data,
while regulators and each bank holding company have access to a
greater wealth of information. Nonetheless, we interpret our
results as a cautionary factor in the analysis of capital plans
produced by bank holding companies as part of a stress test
exercise. At the very least, our results highlight that regulators
may find it difficult to explain their judgment of different bank
holding companies to outside observers by relying exclusively on
public data.

\clearpage

\begin{table}
\center
\begin{tabular}{|l|c|c|c|c|c|c|c|c|c|c|}
\hline
&Step 1 &Step 2 &Step 3 &Step 4 &Step 5 &Step 6 &Step 7 &Step 8 &Step 9 &Step 10\\
\hline
1. F. Combination - Yields          &0.107&0.133&0.148&0.168&0.187&0.202&0.213&0.209&0.216&0.223\\
2. Multivariate Regression          &0.118&0.164&0.202&0.250&0.297&0.345&0.387&0.423&0.464&0.502\\
4. PCR                              &0.118&0.167&0.208&0.259&0.307&0.355&0.397&0.432&0.472&0.508\\
6. F. Combination - Observed Factors&0.118&0.169&0.216&0.273&0.336&0.399&0.456&0.509&0.567&0.622\\
7. VAR on Observed Factors          &0.166&0.209&0.263&0.302&0.342&0.375&0.380&0.430&0.451&0.499\\
8. No-Change Forecast               &0.107&0.138&0.154&0.181&0.206&0.228&0.244&0.248&0.271&0.293\\
\hline
\end{tabular}
\end{table}             

\bibliographystyle{chicago}
\bibliography{bibliosearch}
\clearpage



\end{document}
